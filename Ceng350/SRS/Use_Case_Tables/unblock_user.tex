\begin{table}[H]
     \centering
     \begin{tabular}{|l|C|}
         \hline
          \textbf{Use-Case Name} & Unblock User \\
         \hline
          \textbf{Actors} & Admin \\ 
         \hline
          \textbf{Description} & The admin unblocks a user with given IP address. \\ 
         \hline
          \textbf{Data} & IP address of the user\\ 
         \hline
          \textbf{Preconditions} & 
          The admin must be logged in. \newline
          The admin must have given the "manage connections" command beforehand.\\
         \hline
          \textbf{Stimulus} & The admin types "unblock user" command in the admin command-line interface.\\ 
         \hline
          \textbf{Basic Flow} & 
          Step 1 -- The system asks for the IP address of the user to unblock. \newline
          Step 2 -- The admin gives the IP address of the user obtained from the list displayed after "manage connections" command. \newline
          Step 3 -- Server reaches to the database and sets the blockStatus field of the user record with the given IP Address as "unblocked" \\
         \hline
          \textbf{Alternative Flow\#1} & -\\
         \hline
          \textbf{Alternative Flow\#2} & - \\
         \hline
          \textbf{Exception Flow} & If there is no blocked user with the given IP address, the system displays an error message saying that "No blocked users were found". \\
         \hline
          \textbf{Post Conditions} & The user with the given IP address will be able to connect to the server again. \\ 
         \hline
     \end{tabular}
     \caption{Unblock User}
     \label{tab:unblock_user}
 \end{table}