\begin{table}[H]
     \centering
     \begin{tabular}{|l|C|}
         \hline
          \textbf{Use-Case Name} & Log in as Admin \\
         \hline
          \textbf{Actors} & Admin \\ 
         \hline
          \textbf{Description} & The admin connects to server with his/her login key.\\ 
         \hline
          \textbf{Data} & Admin login key\\ 
         \hline
          \textbf{Preconditions} & - \\
         \hline
          \textbf{Stimulus} & The admin sends login request to server using the terminal.\\ 
         \hline
          \textbf{Basic Flow} & 
          Step 1 -- The server receives the login request. \newline
          Step 2 -- The server asks for the admin login key. \newline
          Step 3 -- Maintainer types his/her login key to the terminal. \newline
          Step 4 -- The system reaches to the local database and verifies the given login key.   \newline
          Step 5 -- The system signs in the admin. \\
         \hline
          \textbf{Alternative Flow\#1} & - \\
         \hline
          \textbf{Alternative Flow\#2} & - \\
         \hline
          \textbf{Exception Flow} & If the given key is invalid, the system displays an error message saying that "Given key is invalid". The system saves the report of the failed login attempt and the IP address of the user to the system logs. Then, the flow starts again from Step 2. \\
         \hline
          \textbf{Post Conditions} & The admin command line interface opens. \\ 
         \hline
     \end{tabular}
     \caption{Log in as Admin}
     \label{tab:log_in_as_admin}
 \end{table}